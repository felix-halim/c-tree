\begin{abstract}


We study main-memory indexing tailored for modern applications
that need to do both efficient reads and efficient updates (both arriving at massive amounts and rates).
We show that state-of-the-art main-memory static indexing methods require too much time 
and workload knowledge in order to go through the preparation phase for a new batch of data, i.e., in order to 
create a main-memory tailored tree (or trie) structure or to sort a main-memory resident array.
Adaptive indexing solves this problem by avoiding expensive initialization costs and incrementally refining indexes; however, 
we show that state-of-the-art adaptive indexing methods are not resilient, i.e., they fail to maintain their adaptive properties
when they are phased with massive sequences of queries and updates.
As a result no current method can support efficient reads and writes.

We identify the root of the problem at the rigid design of indexing methods so far.
We observe that for each stage at a workload there is a different indexing design which is the best fit.
In this paper, we present Trimmer, a new main-memory indexing approach that we call transitive indexing.
Trimmer changes each shape as the workload evolves.
Initially, an index is a collection of unordered vectors. 
As more queries arrive, during query processing, Trimmer refines the hot vectors and  starts
inducing order and structure. Gradually, query after query, Trimmer morphs into a Trie structure
for the hot part of the data. 
As a result Trimmer can absorb massive updates while at the same time it provides 
fast lookup times and it does not need any initialization cost.
A detailed experimental analysis on both synthetic and real-life scenarios from the astronomy domain shows that Trimmer
successfully overcomes the limitations of existing adaptive and static  main-memory indexing -- it enables resilient, adaptive 
and interactive data exploration under massive updates.

%As we enter the era of data deluge, there is an increasing pressure 
%to deal with data by quickly reacting to both queries and new data.
%This phenomenon appears increasingly both in sciences and in businesses where in many cases
%we receive new data every few minutes, summing up to the order of Terabytes daily, while
%we still want to maintain the ability to pose queries with good performance all the time. 
%The main characteristic of such workloads is that there is little (if at all) idle time and workload knowledge
%to do any workload analysis and index tuning to achieve good performance.
%
%Such scenarios call for adaptive approaches to continuously match the workload.
%However, as we show, existing adaptive indexing approaches cannot sufficiently deal 
%with the data deluge challenges; they can handle relatively static scenarios, where most data has arrived up-front, but falter as the size of queries issued and updates performed exceeds the volume of the original data.
%
%
%In this paper, we propose Comb, a new adaptive indexing approach geared towards interactive data exploration.
%Similarly to past adaptive indexing, Comb is designed for main-memory column-stores
%and it incrementally and adaptively reacts to new data and queries, continuously refining the indexes
%and continuously improving the query performance with zero preparation and tuning costs. 
%The innovation in the design of Comb, is that it breaks a single column into multiple disjoint pieces
%and each piece is indexed, accessed and updated independently, while data flows from piece to piece in an adaptive way
%when the index needs to be balanced. 
%A detailed experimental analysis on both synthetic and real-life scenarios from the astronomy domain shows that Comb
%successfully overcomes the limitations of existing adaptive indexing -- it enables resilient, adaptive 
%and interactive data exploration.

\end{abstract}



